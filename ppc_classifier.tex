\documentclass[a4paper]{article}

\usepackage{amsmath}

\addtolength{\oddsidemargin}{-0.7in}
\addtolength{\evensidemargin}{-0.7in}
\addtolength{\textwidth}{1.75in}
\addtolength{\topmargin}{0in}
\addtolength{\textheight}{1.75in}

\DeclareMathOperator{\atan}{atan}

\begin{document}

\section*{Computing PPC features}

First, we compute wavelet transform of bipolar EEG. As the result of it, we have a $4$-dim
array of complex numbers $w_{fprt}$, $f = 1 \ldots n_f$, $p = 1 \ldots n_p$, $r = 1 \ldots n_r$,
$t = 1 \ldots n_t$, (frequencies $\times$ bipolar pairs $\times$ trials $\times$ time points).
Trials are divided into successful (recalls) and unsuccessful classes (non-recalls).

For each $f$, $r$, $t$ and pair $(p_1,p_2)$, we compute the phase difference. The phase
difference between two complex numbers $z_1$ and $z_2$ can be defined as their ratio normalized
to the unit circle, i.e.,
\begin{equation}
\label{eq:phase_diff}
\theta(z_1,z_2) \to \frac{z_1/z_2}{|z_1/z_2|} = \frac{\left(\Re(z_1) \Re(z_2) + \Im(z_1) \Im(z_2)\right) + i \left(\Im(z_1) \Re(z_2) - \Re(z_1) \Im(z_2)\right)}{\sqrt{\left(\Re(z_1) \Re(z_2) + \Im(z_1) \Im(z_2)\right)^2 + \left(\Im(z_1) \Re(z_2) - \Re(z_1) \Im(z_2)\right)^2}}.
\end{equation}

{\em Remark:} though this formula looks more complicated than subtracting to angles when
complex numbers are represented in polar coordinate form (powers and phases), computing
the phases as angles themselves and then operating with them further requires computing
trigonometric ($\cos$, $\sin$) and inverse trigonometric ($\atan$) functions, which are
transcendental functions that are much more expensive to compute than algebraic expressions.
So, we will always assume that complex numbers, as well as their phase differences are
represented in the algebraic form, i.e., $z \to \Re(z) + i \Im(z)$ and
$\theta \to \cos\theta + i \sin\theta$.

We will denote the computed phase differences by $\theta_{frt}^{(p_1,p_2)}$. Next, for every
trial the average {\em phase consistency\/} is computed against every other trial from
{\em the same session and the same class\/}. The phase consistency is defined as
{\em cosine of the angle between phase differences\/}. This is equivalent to applying


(\ref{eq:phase_diff}) to phase differences and 


The phase consistency is averaged over
the time window, so we obtain an array
\[a_{fr}^{(p_1,p_2)} = \frac{1}{n_{\rho} n_t} \sum{t=1}^{n_t}  \]

\end{document}
